\documentclass{article}

\usepackage[paper=a4paper,margin=1in]{geometry}
\usepackage{url}
\usepackage{syntax}
\usepackage{xspace}
\grammarindent5em
\def\atom{\texttt{active}\xspace}
\def\passive{\texttt{passive}\xspace}
\def\grouped{\texttt{grouped}\xspace}




\begin{document}
\subsection*{Algorithm Overview}

The basic idea of the algorithm is to ``fold'' an XML annotation into a SVG in
order to create a tactile diagram. The XML annotation is a navigation structure
that describes multiple hierarchical levels and contains speech annotations its
elements.  It is generally part of an XML document that combines it with
scientific data (e.g., chemical data).  The SVG is an ordinary SVG diagram,
however, hierarchically structured with some elements having id attributes that
correspond to elements of the annotation.

The folding process enriches the SVG, by adding title and description elements
for the overall SVG as well as to the drawn elements. In addition, it introduces
invisible regions by combining multiple drawn elements together. These are elements

E.g., a Benzene ring is composed of six bonds (some of which are double bonds),
which are actual drawn elements in the SVG. The ring itself is simulated by
adding an invisible polygon to the SVG that contains the single bonds but lies
behind them (in terms of z layer). This invisible element furnishes title and
description for the Benzene ring.

We have effectively three types of annotation elements. The motivation behind
some of these elements is best observed with an example, e.g.,
\url{https://progressiveaccess.com/chemistry/generic.html?mole=data/aspirin-enr}.
\begin{description}
\item[active] Single elements that form the base level of the navigation
  structure.  Some active elements do not correspond to a drawn element in the
  SVG. They have to be computed from their neighbouring passive elements.  E.g.,
  Carbon atoms in a chemical molecule can be given by the junction of two or
  more bonds.
\item[passive] Single elements that can be attached to active elements but are
  not interesting enough to be navigatable in their own right in online
  navigation. However, for tactile diagrams they need to be explorable, i.e.,
  get their own title and description.
  
  They always correspond to existing, drawn elements in the SVG.
\item[grouped] Elements that form higher level nagivation elements. They
  correspond to compound structures, made up of active, passive or other grouped
  elements. They have initially no correspondence in the SVG and have to be
  created as invisible polygons.
\end{description}


\subsection*{Algorithm Outline}

\begin{tabular}{ll}
  \textbf{Input:} & Structured SVG \\
  & XML markup \\
  \textbf{Output:} & Audio-tactile SVG
\end{tabular}

\textbf{Cases}

\begin{enumerate}
\item For every \atom and \passive element. If SVG element with corresponding id
  exists, add speech to SVG element. If necessary:
  \begin{enumerate}
  \item embed elements into a new SVG group element,
  \item put an invisible bounding polygon around thin lines.
  \end{enumerate}
\item For every \atom element that does not exists
  \begin{enumerate}
  \item Compute a bounding box by considering the internal neighbours (i.e., for
    take the via elements of neighbour elements with location attribute
    ``internal'').
  \item Create \atom into invisible rectangle with the computed bounding box.
  \item Add speech to new rectangle.
  \end{enumerate}
\item For each \grouped element with parent elements
  \begin{enumerate}
  \item Compute a bounding polygon by combining the SVG bounding boxes of its components.
  \item Add an invisible polygon for the computed bounding polygon.
  \item Add speech to new polygon.
  \end{enumerate}
\item For the \grouped element without parent (top-most element) add speech as
  title and description of the SVG.
\end{enumerate}

Add speech means add \texttt{speech} attribute as SVG title element and \texttt{speech2} attribute (if it exists) as SVG descr element.

\newpage

\subsection*{Grammar}

The following is a grammar outline in extended BNF for the annotation elements.
\vspace{.5cm}
\hrule

\begin{grammar}\setlength{\parskip}{5pt}
  <diagram> ::= `<' <type> `>' <data>* <annotations>  `</' type `>'

  <data> ::= \ldots can be ignored \ldots
  
  <type>  ::= `histogram'
  \alt `circuit'
  \alt `molecule'
  \alt \ldots
  
  
  <annotations> ::= `<annotations>' <annotation>* <messages>* `</annotations>'

  <annotation> ::= `<annotation speech="' <speech> `" speech2="' <speech> `">'
  <element> <position> <parents> <children> <component> <neighbours> `</annotation>'

  <element> ::= <active>
  \alt <passive>
  \alt <grouped>

  <active> ::=  `<active>' <id> `</active>'

  <passive> ::= `<passive>' <id> `</passive>'
  
  <grouped> ::= `<grouped>' <id> `</grouped>'

  <position> ::= `<position>' <number> `</position>'

  <parents> ::= `<parents>' <grouped>? `</parents>'

  <children> ::= `<children>' <element>* `</children>'

  <component> ::= `<component>' <element>* `</component>'

  <neighbours> ::= `<neighbours>' <neighbour>* `</neighbours>'

  <neighbour> ::= `<neighbour speech="' <speech> `" speech2="' <speech> `" location="' <location> `">' <element> <via>+ `</neighbour>'

  <location> = internal | external
  
  <via> ::= `<via>' <passive> <position> `</via>'

  <id> ::= <alpha> <id>* | <digit> <id>* | <symbol> <id>*

  <number> ::= <digit> | <digit> <number>

  <digit> ::= 0 | 1 | 2 | 3 | 4 | 5 | 6 | 7 | 8 | 9

  <alpha> ::= a-z

  <Alpha> ::= A-Z

  <symbol> :: = - | _

  <speech> ::=  <string> | <msgid>
  
  <string> ::= `Unicode'*

  <messages> ::= `<language>' <language> `</language>' <message>*

  <message> ::= `<message msg="' <msgid> `">' <string> `</message>'

  <msgid> ::= <Alpha> <msgid>* | <symbol> <msgid>* | <number> <msgid>*
  
  <language> ::= iso-639-1 language code
  
\end{grammar}
\hrule

\subsubsection*{Notes:}

\begin{enumerate}
\item The grammar is defined in two namespaces:
\begin{itemize}
\item \emph{data} elements are definied in a namespace suitable for the expressed data.
\item \emph{annotations} elements are defined in the \texttt{sre}
  namespaces. Hence elements are usually of the form \texttt{sre:annotations}
  etc.
\end{itemize}
\item The \texttt{component} element for active element annotations can only
  contain \texttt{passive} elements.
\item Multi-linguality is achieved via the message elements.
\begin{itemize}
\item In case \texttt{messages} elements are present, the \texttt{<speech>} elements are \texttt{<id>}s
\item If there are no \texttt{messages} elements, the attributes \texttt{speech}
  and \texttt{speech2} contain each contain a <string> only.
\end{itemize}
\item Special case of molecule diagrams:
\begin{itemize}
\item \texttt{active}, \texttt{bond}, \texttt{grouped}, are called
  \texttt{atom}, \texttt{bond}, \texttt{atomSet}, respectively.
\end{itemize}
\end{enumerate}

\end{document}
